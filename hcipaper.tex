\documentclass{sigchi}

% Use this command to override the default ACM copyright statement (e.g. for preprints). 
% Consult the conference website for the camera-ready copyright statement.


%% EXAMPLE BEGIN -- HOW TO OVERRIDE THE DEFAULT COPYRIGHT STRIP -- (July 22, 2013 - Paul Baumann)
% \toappear{Permission to make digital or hard copies of all or part of this work for personal or classroom use is 	granted without fee provided that copies are not made or distributed for profit or commercial advantage and that copies bear this notice and the full citation on the first page. Copyrights for components of this work owned by others than ACM must be honored. Abstracting with credit is permitted. To copy otherwise, or republish, to post on servers or to redistribute to lists, requires prior specific permission and/or a fee. Request permissions from permissions@acm.org. \\
% {\emph{CHI'14}}, April 26--May 1, 2014, Toronto, Canada. \\
% Copyright \copyright~2014 ACM ISBN/14/04...\$15.00. \\
% DOI string from ACM form confirmation}
%% EXAMPLE END -- HOW TO OVERRIDE THE DEFAULT COPYRIGHT STRIP -- (July 22, 2013 - Paul Baumann)


% Arabic page numbers for submission. 
% Remove this line to eliminate page numbers for the camera ready copy
% \pagenumbering{arabic}


% Load basic packages
\usepackage{balance}  % to better equalize the last page
\usepackage{graphics} % for EPS, load graphicx instead
\usepackage{times}    % comment if you want LaTeX's default font
\usepackage{url}      % llt: nicely formatted URLs
\usepackage{subcaption}
\usepackage{float}
% llt: Define a global style for URLs, rather that the default one
\makeatletter
\def\url@leostyle{%
  \@ifundefined{selectfont}{\def\UrlFont{\sf}}{\def\UrlFont{\small\bf\ttfamily}}}
\makeatother
\urlstyle{leo}


% To make various LaTeX processors do the right thing with page size.
\def\pprw{8.5in}
\def\pprh{11in}
\special{papersize=\pprw,\pprh}
\setlength{\paperwidth}{\pprw}
\setlength{\paperheight}{\pprh}
\setlength{\pdfpagewidth}{\pprw}
\setlength{\pdfpageheight}{\pprh}

% Make sure hyperref comes last of your loaded packages, 
% to give it a fighting chance of not being over-written, 
% since its job is to redefine many LaTeX commands.
\usepackage[pdftex]{hyperref}
\hypersetup{
pdftitle={SIGCHI Conference Proceedings Format},
pdfauthor={LaTeX},
pdfkeywords={HCI, Cooperative, Spaceship, Game},
bookmarksnumbered,
pdfstartview={FitH},
colorlinks,
citecolor=black,
filecolor=black,
linkcolor=black,
urlcolor=black,
breaklinks=true,
}

% create a shortcut to typeset table headings
\newcommand\tabhead[1]{\small\textbf{#1}}


% End of preamble. Here it comes the document.
\begin{document}

\title{SpaceRescue: A communicative Game for Interactive Floors}

\numberofauthors{1}
\author{
  \alignauthor Daniel Birnstiel, Patrick Kuhn, Fabian Paul, Lennard Wolf\\
      \medskip
    \affaddr{Hasso Plattner Institute}\\
    \affaddr{Prof.-Dr.-Helmert-Str. 2-3}\\
    \affaddr{14482 Potsdam, Germany}\\
    \email{$ \{ $ daniel.birnstiel, patrick.kuhn, fabian.paul, lennard.wolf $ \} $@student.hpi.de }\\
}

\maketitle

\begin{abstract}
\vspace{1mm}
After looking upon the possibilities of interactive floors and taking into account the immense demand for video games today, we developed \textit{SpaceRescue}. \textit{SpaceRescue} is designed as a fun prototype for communicative, interactive floor based games and demonstrates, how playing games at home can once again involve moving the entire body
again - and not just your thumbs.


The two to three players' goal is the joint reaching of levels by performing tasks that are randomly given to each player. These tasks  then have to be carried out by the player himself or his partner, if the task-subject is not part of that player's controls stack. 

\end{abstract}

\keywords{
	SpaceRescue; Communication; Interactive Floors; Body Movement
}

\section{Introduction}
\vspace{1mm}
With the advent of the \textit{Internet of Things} and thus the rising digital interaction with everything around us, floors might soon become intelligent entities just like our phones are today. 

But next to all the productive things we can now do with our devices, we also want to integrate them into our leisure time. Just like touch screens revolutionised the way we play games, interactive floors would again push the boundaries of the way we think about enjoying ourselves through games.

To understand what players want from a game, we interviewed J\"org Friedrich, a professional game designer at AAA game development studio \textit{Yager}, Willy Scheibel, a game development lecturer at \textit{HPI}, as well as Theresa Zobel, a representative of our target group, the casual gamer. From these interviews we learned that:

\begin{figure}[H]
\centering
\includegraphics[width=0.99\columnwidth]{walkthrough/totale2}
\caption{The Game in action.}
\label{fig:figure1}
\end{figure}

\begin{itemize}
\item The game mechanics are the very most important part in the success of a game and should mostly be based on already proven concepts.
\item Today's players don not want to read complex manuals and thus want to understand the game right away.
\item Gamers always want to be challenged, but not to the point where they get frustrated.
\item Next to the challenge, gamers also need a purpose for their actions. This might involve a story or the immersion in a virtual world.
\item Pleasing graphics are nice to have but not overly important, especially not photorealism.
\item According to sales stats, most gamers today like to play cooperative games rather than single player titles.
\end{itemize}

\vspace{4mm}

\begin{figure}[h]
\centering
\includegraphics[width=0.8\columnwidth]{leute}
\caption{From left to right: Casual Gamer Theresa Zobel, game development lecturer Willy Scheibel, the research group with professional game designer  J\"org Friedrich (2nd from the right)}
\label{fig:beamingArea}
\end{figure}
\pagebreak
This is why we started out contemplating different mini games and at some point even considered creating a mini game collection. But what we saw as the main advantage of an interactive floor based game was the fact that the player might not be alone, but rather be in the same room with others and cooperate. 

Therefore we chose to concentrate on three main goals for our project: \textbf{Communication}, \textbf{Cooperation} and \textbf{Discoverability}.





The game is set on a spaceship, which is on the edge of destruction and the players on earth are the astronauts to save the ship. They have to get on the beaming platforms to beam onto the spaceship, where they get their tasks. These tasks involve flipping switches and pressing buttons, which make up the control bridge of the ship. 

Since these widgets are not always part of the player's control stack, he will often have to tell his partner to do it for him instead. But a task is only active for a limited amount of time and if a task is not performed in time, it's game over. \newline
After a certain number of finished tasks, ladders will appear that lead the players to a new and harder level, where all controls will change and the game goes on. Winning as such is \textit{not} possible, the motivation is rather staying alive for  \textit{as long as possible}.
\section{Walkthrough}
\vspace{1mm}
In our sample scenario two users Anne and Bernd want to play a game of \textit{Cooperative Spaceship Rescue} with the goal to reach the second Level and abort the game with pushing on eject. As the users enter the floor, they will start out in the \textit{Beaming Area}, as shown in Figure 8.

\subsection{The Beaming Area}
\vspace{3mm}
At the beginning both of them see a rotating text in the center of the floor. The text introduce Bernd and Anna to choose one platform. After taking their position in front of Anna and Bernd appear two buttons with a ready label. These buttons suggest the players to tap. Immediately after pushing the buttons they will change into label screens. Bernd can now start the game with a start button. "Start" signifies a screen changing event.


\begin{figure}[h]
\centering
\includegraphics[width=0.8\columnwidth]{beamingArea}
\caption{The Beaming Area from the bird's perspective.}
\label{fig:beamingArea}
\end{figure}


\begin{figure}[H]
\centering
\includegraphics[width=0.9\columnwidth]{walkthrough/klein/beamingarea1+2a}
\caption{Steps 1 and 2a}
\label{fig:gamingArea}
\end{figure}

Anna and Bernd mention behind each is a highlighted task panel so they turn around and read their tasks. During this a timer which is placed in the lower right corner of the panel starts ticking. Anna has the task "Activate the flux cannon" but she can not see anything which will match it on her side. She has to communicate with Bernd to find the widgets for completing the task. After a task has been performed before the corresponding timer has countered to zero, a new task will be generated and timer is reset. Bernd and Anna complete some tasks which will lead them to the next level where Anna push the eject button to finish and abort the game.

\begin{figure}[H]
\centering
\includegraphics[width=0.9\columnwidth]{walkthrough/klein/beamingarea2b+3}
\caption{Steps 2b and 3}
\label{fig:gamingArea}
\end{figure}




\subsection{The Gaming Area}



\begin{figure}[H]
\centering
\includegraphics[width=0.9\columnwidth]{walkthrough/gamingarea}
\caption{The whole \textit{gaming area} from the bird's perspective.}
\label{fig:beamingArea}
\end{figure}
\vspace{2mm}
\begin{enumerate}
\item Anne and Bernd mention the two highlighted task panels behind them and turn around.
\item There, each player reads his instructions and sees their timer counting down.


\begin{figure}[H]
\centering
\includegraphics[width=0.99\columnwidth]{walkthrough/klein/gamingarea3+4}
\caption{Steps 3 and 4}
\label{fig:gamingArea}
\end{figure}

\item Anne gets task X (for example "Activate Flux Cannon" as shown in Figure 6).
\item Bernd gets task Y (for example "Disable Endo-Sensors" as shown in Figure 6).
\item Anne and Bernd try to perform their task by communicating or finding their widget.
\item After a task has been performed before the corresponding timer has counted to zero, a new task is be generated and the timer is reset.

\begin{figure}[H]
\centering
\includegraphics[width=0.99\columnwidth]{walkthrough/klein/gamingarea5+6}
\caption{Steps 5 and 6}
\label{fig:gamingArea}
\end{figure}
\item Repeat steps 3. to 6. until a new level is generated and the players have reached their goal.
\end{enumerate}


\section{Design}
\vspace{1mm}
In this section we will discuss the design choices we made to achieve our goals for \textbf{communication}, \textbf{cooperation} and \textbf{discoverability}.

\subsection{Cooperation as a Paradigm}
\vspace{1mm}
Our first idea was to create an application which would consist of various \textit{minigames}, which could be successively played against each other. The problem with this approach was, as Willy Scheibel pointed out in our contextual inquiry, that this approach would only allow users to interact with each other to certain degree. Playing by taking turns would not actually require the other player to be there, thus the whole concept of being together in a room vanishes. 

Therefore we wanted to integrate cooperation as a paradigm for our game design, by assigning tasks to users, who cannot accomplish them.

\subsection{Separating the floor into two areas to enforce communication}
\vspace{1mm}
One of our basic game principles is communication between the users. To avoid the users from trying to accomplish the task by themselves rather than telling the other teammate what to do, we separated the floor with the graphic of rifts, where one can see the background of stars. When testing the design with users, we observed that some of them simply jumped over the rifts, because they perceived them as normal obstacles. To prevent this from happening, we added yellow \textit{"do not cross"}-tapes on both sides of the rifts. This ensures, that each user only operates his own controls.

\begin{figure}[H]
\centering
\includegraphics[width=0.99\columnwidth]{dangerarea}
\caption{The \textit{danger area}, which seperates the two control rooms}
\label{fig:gamingArea}
\end{figure}

\subsection{Avoiding roles as a Game Mechanic}
\vspace{1mm}

Originally we came up with the idea of having different roles for each user, so that each would only be able to carry out certain tasks. As an example, the role of the \textit{captain} was assigned to the first person entering a beaming platform. That user was then the one able to start the game and they would also have special rights to certain widgets and so on. In our paper prototyping process, most of the testers asked about the meaning of those roles and having to explain it to them in more than a sentence showed how this concept was hardly discoverable and would just \textit{mode the user in}.

We decided to drop the idea of roles as an explanation for the users abilities to accomplish certain tasks. The seperation of tasks is now only created by the spatial placement of the buttons like we explained in our previous section.  

\subsection{Explaining System Status with metaphors}
\vspace{1mm}
As we started designing our game, we used menus and text fields to let the user start a game or change a level. This approach seemed to work out quite well in paper prototyping, since the change of level for example would require some time to reconstruct the floor anyway. But as we started testing our application on the floor the change of status, especially the switching between two levels did not appear natural. As a result we had to the change the core structure of our design to more modular areas.

The switching between levels was changed from only recreating the widgets to having an own scene were two ladder icons are shown, which have to be stepped on to reach the next level. This makes the user clear why the change of appearance of the floor happens. Another example of this approach are the beaming platforms in our start screen that are used to enter the level. This allows us to explain everything what happens in the world in terms of our metaphors. 

\begin{figure}[H]
\centering
\includegraphics[width=0.8\columnwidth]{ladder}
\caption{The message explaining the status with a ladder next to it}
\label{fig:gamingArea}
\end{figure}


\subsection{Standing in a defined Area as Login Mechanism}
\vspace{1mm}
In typical PC or console games users log in by typing their user name and select start in the menu. Initially we thought about applying this concept to the floor by having the users register to the system with an on-floor keyboard and then letting them log in every time they enter the floor. During \textit{paper prototyping} we encountered the issue that it is really tedious for the users to type their names, since tapping on small buttons requires precision and having the buttons spread makes them hard to use because it would be necessary to walk over them to get to the destination. 
\begin{figure}[H]
\centering
\includegraphics[width=0.99\columnwidth]{loginarea}
\caption{Two users logging in through the \textit{Beaming Platforms}}
\label{fig:beamingArea}
\end{figure}
We decided to create areas in which the users have to stand to start the game. The paper prototyping and the heuristic evaluation showed that this was easily understandable, especially because of the conceptual model of the \textit{beaming area} of spaceships from \textit{Star Trek}. 




\subsection{Having a distinct Exit Button}
\vspace{1mm}
When we faced the issue of how the game should be terminated, our initial solution was to just let the users walk off the floor and wait until the timer would run out. But during the heuristic evaluation, almost all of our testers asked us how they could end the game. This showed us that having no visual representation to use some functionality of the system is not at all discoverable. 

To solve this problem we introduced a distinct exit button, which can be pressed to ensure a controlled ending of the game. This is important because now the user has full control over the system at any time. 


\begin{figure}[H]
\centering
\includegraphics[width=0.99\columnwidth]{ejectbutton}
\caption{On the left the concept of the message if all users have been \textit{Eject} button that pauses the game}
\label{fig:gamingArea}
\end{figure}


\begin{figure}[H]
\centering
\includegraphics[width=0.99\columnwidth]{ejecting}
\caption{After pushing the \textit{Eject} button, a siren starts off}
\label{fig:gamingArea}
\end{figure}

\section{Conclusion}
\vspace{1mm}

Creating a game for a completely new platform turned out to be a bigger challenge than first anticipated, since most classic gaming concepts were not applicable anymore. But talking to both gamers and the professionals in the field gave us great insights into what makes a good game. After applying these insights to the design of \textit{SpaceRescue}, it now shows the possibilities for communication and involvement of the whole body of interactive floor based games. It is important to not just mindlessly use up the provided space of the floor, but to make it a part of the system and this was eventually the key to making SpaceRescue enjoyable.\newline





\balance


\end{document}
